\documentclass[12p]{article}

\usepackage{geometry} % Required to change the page size to A4
\geometry{a4paper} % Set the page size to be A4 as opposed to the default US Letter

\usepackage[utf8]{inputenc}
\usepackage{graphicx}
\usepackage{float} % Allows putting an [H] in \begin{figure} to specify the exact location of the figure
\usepackage{wrapfig} % Allows in-line images such as the example fish picture
\usepackage[nopar]{lipsum} % Used for inserting dummy 'Lorem ipsum' text into the template
\usepackage{fancyhdr}
\usepackage[parfill]{parskip} % Makes sure to put line breaks in between paragraphs and have no indentation
\usepackage{dirtytalk} % Used for quotations (\say{quote})
\usepackage[toc,page]{appendix}
\usepackage{caption}
\usepackage{subcaption}
\usepackage{url}

\usepackage[
    style=numeric,
    sorting=none
    ]{biblatex}
\addbibresource{references.bib}

\linespread{1.2} % Line spacing
\setlength\parindent{0pt} % Globally suppress indentation
 
\graphicspath{{pics/}} % Specifies the directory where pictures are stored

\pagestyle{fancy} % Use this, if a header on each page with the section title and page number is wanted
\fancyhf{} % Removes all headers and footers, comment this to show page number at the bottom of each page again
\fancyhead[L]{\rightmark} % Sets the section title on the left side of the header
\fancyhead[R]{\thepage} % Sets the page number on the right side of the header

\newcommand{\HRule}{\rule{\linewidth}{0.5mm}} % Defines a new command for horizontal lines
\newcommand{\SlimHRule}{\rule{\linewidth}{0.25mm}} % Defines a new command for horizontal lines

%----------------------------------------------------------------------------------------

\begin{document}

%----------------------------------------------------------------------------------------
%	TITLE PAGE
%----------------------------------------------------------------------------------------

\begin{titlepage}
    
    \center
    
    %------------------------------------------------
	%	Logo
	%------------------------------------------------
	
	\includegraphics[width=0.2\textwidth]{pics/AAU_Logo.png}\\[1cm]
    
    %------------------------------------------------
	%	Headings
	%------------------------------------------------
	
	\textsc{\LARGE Aalborg University Copenhagen}\\[1.5cm]
	
	\textsc{\Large P1 Project}\\[0.5cm]
	
	\textsc{\large Group 007}\\[0.5cm]
	
	\textsc{\large IT, Communication and New Media}\\[0.5cm]
	
	
	%------------------------------------------------
	%	Title
	%------------------------------------------------
	
	\HRule\\[0.4cm]
	
	{\huge\bfseries Gangster Squirrel}\\[0.4cm]

	\HRule\\[1.5cm]
	
	%------------------------------------------------
	%	Author(s) and Supervisor(s)
	%------------------------------------------------
	
	\begin{minipage}{0.4\textwidth}
    \begin{flushleft} \large
    \emph{Authors}\\
        Ludvig Alexander \textsc{Brüchmann} \\
        Muheb \textsc{Khan} \\
    	Rehan \textsc{Mir} \\
    	Johannes \textsc{Mols} \\
    	Martin \textsc{Sander} \\
    	Agata \textsc{Surmacz} \\
    	Boris \textsc{Yordanov} \\
    \end{flushleft}
    \end{minipage}
    ~
    \begin{minipage}{0.4\textwidth}
    \begin{flushright} \large
    \emph{Study Numbers} \\
        20174692 \\
        - \\
        20174693 \\
        20174921 \\
        - \\
        20173800 \\
        - \\
    \end{flushright}
    \end{minipage}\\[0.5cm]
    
    %------------------------------------------------
    
    \begin{minipage}{0.4\textwidth}
    \begin{flushleft} \large
    \emph{Supervisors}\\
        Morten \textsc{Falch} \\
        Lene Tolstrup \textsc{Sørensen} \\
    \end{flushleft}
    \end{minipage}
    ~
    \begin{minipage}{0.4\textwidth}
    \begin{flushright} \large
    \end{flushright}
    \end{minipage}\\[0.5cm]

	%------------------------------------------------
	%	Date
	%------------------------------------------------
	
	\vfill\vfill\vfill % Position the date 3/4 down the remaining page
	
	{\large\today} % Date, change the \today to a set date if you want to be precise
	
    
\end{titlepage}

%----------------------------------------------------------------------------------------
%	SYNOPSIS / ABSTRACT
%----------------------------------------------------------------------------------------

\begin{abstract}
\thispagestyle{plain} %Sets the page style for this specific page to plain, to remove the header and show the page number at the bottom
    This is our Abstract, my dudes.

\end{abstract}

\newpage

%----------------------------------------------------------------------------------------
%	TABLE OF CONTENTS
%----------------------------------------------------------------------------------------

\tableofcontents % Include a table of contents
\thispagestyle{plain} %Sets the page style for this specific page to plain, to remove the header and show the page number at the bottom

\newpage % Begins the essay on a new page instead of on the same page as the table of contents 

%----------------------------------------------------------------------------------------
%	INTRODUCTION
%----------------------------------------------------------------------------------------

\section{Introduction}

Text

\begin{center}
    \vspace{1em}
    \SlimHRule\\[0.1cm]
    \Large{Our problem formulation}
    \SlimHRule\\[0.1cm]
    \vspace{1em}
\end{center}

Text \medskip

Text

%----------------------------------------------------------------------------------------
%	METHODOLOGY
%----------------------------------------------------------------------------------------

\newpage
\section{Methodology} \label{Methodology}

\textbf{Introduction}

Text. \medskip

\textbf{Planning}

Text. \medskip

\textbf{PEST analysis (Section \ref{PEST})}

Text. \medskip

\textbf{Stakeholder analysis (Section \ref{Stakeholder_Analysis})}

Text. \medskip

\textbf{Market Segmentation (Section \ref{MarketSegmentation})}

Text. \medskip

\textbf{Blue Ocean - Red Ocean (Section \ref{BlueOceanRedOcean})}

Text. \medskip

\textbf{SWOT analysis (Section \ref{SWOT})}

Text. \medskip

\textbf{The four P's (Section \ref{TheFourPs})}

Text.  \medskip

\textbf{Marketing strategy (Section \ref{MarketingStrategy})}

Text.

%----------------------------------------------------------------------------------------
%	STATE OF THE ART
%----------------------------------------------------------------------------------------

\newpage
\section{State of the art}

Introduction on state of the art and how we searched and found our competitors.

%------------------------------------------------

\subsection{Tallowmere} \label{StateOfTheArt_Tallowmere}

\begin{figure}[h]
    \center
    \includegraphics[width=1\textwidth]{StateOfTheArtScreenshots/tallowmere}
    \label{StateOfTheArt_Screenshots_Tallowmere}
    \caption{Screenshot of the game Tallowmere \cite{TallowmereScreenshot}}
\end{figure}

A 2D action platformer with an emphasis on executing your hero's jumping, shield-blocking, and weapon-attacking at the right moments as you move through a procedurally-generated dungeon. The game is available for:

\emph{Windows XP SP2 or newer, macOS 10.8 or newer, SteamOS, Ubuntu 12.04, Android 4.4 or newer, iOS 8.1 or newer, Wii U}

Features:

\begin{itemize}
    \item Randomly-generated rooms, with each room bigger than the last
    \item Blood splats, particle effects and sounds
    \item Skill-based gameplay, with no ammo, mana, or stamina gauges
    \item Traps and obstacles to dodge and avoid
    \item Special boss and event rooms
    \item Different weapons, shields and garments the hero can equip
    \item Infinite jumping
    \item Single Player and local co-op (up to 4 players)
    \item Permanent death, with a typical run lasting from 1 minute to 2 hours depending on skill
\end{itemize}

%------------------------------------------------

\subsection{Cave Story Plus}

\begin{figure}[h]
    \center
    \includegraphics[width=1\textwidth]{StateOfTheArtScreenshots/cave_story_plus}
    \label{StateOfTheArt_Screenshots_CaveStoryPlus}
    \caption{Screenshot of the game Cave Story+ \cite{CaveStoryPlusScreenshot}}
\end{figure}

Cave Story+ is a Cave Story remake for PC, Mac and Nintendo Switch developed by Nicalis. It was released on Steam on November 22, 2011. The game features remastered graphics and music as well as several new game modes, of which one is only made exclusively to the Nintendo Switch version \cite{CaveStoryPlusWiki}.

This game offers a unique storyline based on Mysterious themes and it takes you to an amazing world of Mimiga, the land of the Free Running Rabbits named as Mimigas \cite{CaveStoryPlusMoreGamesLike}.

Features:

\begin{itemize}
    \item Background music
    \item Over 20 boss battle games
    \item A variety of weaponry
    \item Four different game endings
    \item Hard mode for veterans
    \item Multiple versions available
    \item Metroid inspired setting
    \item Simple storyline
\end{itemize}

%------------------------------------------------

\subsection{Stardew Valley}

\begin{figure}[h]
    \center
    \includegraphics[width=1\textwidth]{StateOfTheArtScreenshots/stardew_valley}
    \label{StateOfTheArt_Screenshots_StardewValley}
    \caption{Screenshot of the game Stardew Valley \cite{StardewValleyScreenshot}}
\end{figure}

Stardew Valley is a 2D top-down indie farming simulation, which features an endless mine, where the player can fight, gather resources for the character and his farm and explore the different levels. The mines consist of an endless amount of procedurally generated levels, with checkpoints after every five levels. These checkpoints can be reached by an elevator on the first level in the mines. Each level consists of several elements, which are mainly gatherable stones and enemies, and also enemies. Furthermore, some levels have special elements, which are quicksand, minecarts, chests, and more. To reach the next level, the player has to find a ladder down, which can be found under a random stone or ore. They can either be destroyed by a pickaxe or bomb. This mine system is only a small part of the overall game but offers a fun and highly replayable experience due to the infinite amounts of levels.

Features:

\begin{itemize}
    \item Top-down rather than a sidescrolling game
    \item Procedurally generated levels
    \item Different themes and enemies for different level regions
    \item Inventory system
    \item Health and energy system
    \item Upgrading of weapons and armor
    \item Resource gathering (for overall game)
    \item Simple 8-bit graphics with customizable character
\end{itemize}

%------------------------------------------------

\subsection{Rogue Legacy}

\begin{figure}[h]
    \center
    \includegraphics[width=1\textwidth]{StateOfTheArtScreenshots/rogue_legacy}
    \label{StateOfTheArt_Screenshots_RogueLegacy}
    \caption{Screenshot of the game Rogue Legacy \cite{RogueLegacyScreenshot}}
\end{figure}

Rogue Legacy is a 2D game, which was released in 2013 developed by Cellar Door Games. The game has been released for different platforms such as Microsoft Windows, Linux, OS X, PlayStation 3, PlayStation 4, PlayStation Vita and Xbox One \cite{RogueLegacyWiki}. The game is about exploring a randomly generated dungeon and collecting gold. The player(character) has default abilities such as jump and slash with a sword and secondary abilities such as magic attacks by using mana. The player has to defeat a boss in order to go to the next level \cite{RogueLegacyReview}.

Features: \cite{RogueLegacySteam}

\begin{itemize}
    \item Single player
    \item With sword, mana and life gauge to worry about
    \item Traps and obstacles to dodge and avoid
    \item More than 8 classes to choose from
    \item Over 60 different enemies
    \item Powerful weaponry and armor
    \item Tons of secret areas
    \item Boss in every stage
    \item Jump, run and dash
\end{itemize}

%------------------------------------------------

\subsection{Mega Miner}

\begin{figure}[h]
    \center
    \includegraphics[width=0.8\textwidth]{StateOfTheArtScreenshots/mega_miner}
    \label{StateOfTheArt_Screenshots_MegaMiner}
    \caption{Screenshot of the game Mega Miner \cite{MegaMinerScreenshot}}
\end{figure}

Mega miner is a 2D game developed by Armor Games. The main task in the game is to drill a mine and while this process collects gold, silver, coal and other minerals. Then the player (character) can sell it for cash and upgrade his mining machine by new coolers etc. This game is qualified as a strategy game.

Features:

\begin{itemize}
    \item Single player
    \item Gold, silver, coal and other minerals to collect
    \item Rocks as obstacles to avoid
    \item One map
    \item No enemies
    \item Possibility to upgrade character (drill)
\end{itemize}

%------------------------------------------------

\subsection{Spelunky}

\begin{figure}[h]
    \center
    \includegraphics[width=1\textwidth]{StateOfTheArtScreenshots/spelunky}
    \label{StateOfTheArt_Screenshots_Spelunky}
    \caption{Screenshot of the game Spelunky \cite{SpelunkyScreenshot}}
\end{figure}

Spelunky is a 2D cave exploration/treasure-hunting game inspired by classic platform games such as Rayman, Super Mario, and Sonic the Hedgehog. One of the crucial features that differentiate this game to the classic versions, is that the game is top-down rather than sidescrolling, which leaves the player with a completely different feel. With an inventory full of bombs, ropes, and your own whip, the aim is to creatively maneuver through the fully-destructible ancient caves. In order to complete levels, you have to search deep in the underground for a door, while encountering a variety of enemies, as well as traps and treasure. It's a challenging game that will likely test your patience since you will die continuously. But with many deaths comes experience, and eventually, as you beat the level, the frustration is overtaken by a great feeling of satisfaction.

\textbf{Pros and Cons:}

\textbf{Pro:} Skill based gameplay. \emph{In Spelunky the player has everything needed from the start so it is only the skill level of the player that determines whether you advance or not.}

\textbf{Pro:} Roguelike genre allow for quick sessions. \emph{Since the gameplay is so quick, the player will die quite often which means that the sessions tend to be quite short. This makes for a game that can be played in quick bursts throughout the day.}

\textbf{Con:} No check-point system. \emph{You can't save the game and you can't go back to a level with your previous equipment.}

\textbf{Con:} Hard to master. \emph{The game is difficult to the point of frustration.}

Features:

\begin{itemize}
    \item Top-down rather than a sidescrolling game
    \item Skill-based gameplay
    \item Single player and local co-op (up to 4 players)
    \item Traps and enemies to dodge and avoid
    \item Inventory system primarily with bombs and ropes
    \item Money collecting for the shopkeeper in-between levels
    \item Full maneuverability with infinite jumping and running
\end{itemize}

%------------------------------------------------

\subsection{Nuclear Throne}

%------------------------------------------------

\subsection{Conclusion}
Conclusion.

%----------------------------------------------------------------------------------------
%	MARKET ANALYSIS
%----------------------------------------------------------------------------------------

\newpage
\section{Market analysis} \label{MarketAnalysis}

%------------------------------------------------

\subsection{Analysis of external factors (PEST)} \label{PEST}

\textbf{Political factors}

Text. \medskip

\textbf{Economic factors}

Text. \medskip

\textbf{Socio-Cultural factors}

Text. \medskip

\textbf{Technological factors}

Text. \medskip

\textbf{Conclusion}

Conclusion.

\newpage

%------------------------------------------------

\subsection{Stakeholder analysis} \label{Stakeholder_Analysis}

Text.

\newpage

%------------------------------------------------

\subsection{Market segmentation} \label{MarketSegmentation}

Text.

\newpage

%------------------------------------------------

\subsection{Blue ocean - Red ocean} \label{BlueOceanRedOcean}

Text.

\newpage

%------------------------------------------------

\subsection{SWOT analysis} \label{SWOT}

I left this in as a template for this kind of table, it took me a lot of time to figure that one out.

\begin{table}[H]
  \centering
    \begin{tabular}{|p{0.5\textwidth}|p{0.5\textwidth}|}
    \hline
    \multicolumn{1}{|c|}{\textbf{Strengths}} & \multicolumn{1}{c|}{\textbf{Weaknesses}} \\
    \hline
    {• We can respond very quickly to our customers in Copenhagen and it's areas when they order on our platform\newline{}• Our prices are low and competitive compared to the market\newline{}• We know what our target customer is looking for if he chooses us, and therefore we can answer his needs very rapidly\newline{}• Our products are of certified good quality} & • Our company has little market presence or reputation\newline{}• We have a small staff, with limited skill base in many areas\newline{}• Higher price point compared to supermarkets since the service is included\newline{}• Delivery fees are higher by night time \\
    \hline
    \end{tabular}
  \label{tab:swot_one}
\end{table}

\begin{table}[H]
  \centering
    \begin{tabular}{|p{0.5\textwidth}|p{0.5\textwidth}|}
    \hline
    \multicolumn{1}{|c|}{\textbf{Opportunities}} & \multicolumn{1}{c|}{\textbf{Threats}} \\
    \hline
    {• Our competitors offer is very limited, so we can expand more over time\newline{}• The sector of alcohol is always growing, which leaves a gap in the market\newline{}• No real competitors in Denmark} & • Selling alcohol license may be hard to obtain\newline{}• Emergence of future competitors\newline{}• Customer choosing other options such as DøgnNetto, or Bilka \\
    \hline
    \end{tabular}
  \label{tab:swot_two}
\end{table}

\newpage

%------------------------------------------------

\subsection{The four P's} \label{TheFourPs}

Text.

\newpage

%------------------------------------------------

\subsection{Marketing strategy} \label{MarketingStrategy}

Text.

%----------------------------------------------------------------------------------------
%	DOCUMENTATION
%----------------------------------------------------------------------------------------

\newpage
\section{Documentation}

\subsection{Choosing the game engine}

%----------------------------------------------------------------------------------------
%	DISCUSSION
%----------------------------------------------------------------------------------------

\newpage
\section{Discussion}

Discussion.

%----------------------------------------------------------------------------------------
%	CONCLUSION
%----------------------------------------------------------------------------------------

\newpage
\section{Conclusion}

Project Conclusion.

%----------------------------------------------------------------------------------------
%	BIBLIOGRAPHY
%----------------------------------------------------------------------------------------

\newpage
\printbibliography[heading=bibintoc,title={References}]

%----------------------------------------------------------------------------------------
%	APPENDIX
%----------------------------------------------------------------------------------------

\newpage
\appendix
\section{Appendix}

%------------------------------------------------

\end{document}